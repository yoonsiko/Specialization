\subsection{Background and motivation}
The world is in a desperately need for a new energy solution. Everything around us is heavily dependent on energy generated on \ce{CO2} emitting sources, such as a coal power plant. Even though the majority of western industry is shifting towards ``The green change'', the rate of the shift is not enough. Europe is especially suffering the consequences of the gas dependent infrastructure as Russia stopped supplying Europe with gas for energy and heating. People in Europe can't afford to heating and food. Inflation has never been higher in Eurozone as for the last decades. Electricity generated from various renewable energy sources such as hydro power, wind mills and solar panels is not enough to replace the huge demand of energy that the current global infrastructure requires. But there is a rising interest and research for an another type of fuel; hydrogen.

Hydrogen is widely used for various purposes around the world and the industry. Hydrogen is also considered to be a fuel source without greenhouse gas emissions as the only emission is water vapour. Even though hydrogen as fuel is without emission when used, the production is still too harmful to the environment due to all the emission and the energy required. There are three ways to produce hydrogen, but the most used method of production around the globe is ``grey hydrogen''. Grey hydrogen is produced via steam-methane reforming, which uses methane and water to produce hydrogen. Grey hydrogen also emits a relatively high amount of \ce{CO2}, which is the main downside of grey hydrogen. Blue hydrogen, which also uses steam-methane reforming, but instead of letting the \ce{CO2} emit into the atmosphere, the $>95\%$ of \ce{CO2} is captured and stored. The last method is green hydrogen, which is produced by splitting water molecule to hydrogen and oxygen via electrolysis.

Even though grey hydrogen is widely used around the world to produce hydrogen, but its more environmental counterpart, blue hydrogen is not used widely. Economy and the extra need for units is the main reason why blue hydrogen is less used than grey hydrogen. Norway is the country with the most carbon capture in its offshore rigs, but with today's technology, it requires the natural gas to be transported to onshore, process the gas, capture the \ce{CO2} and transport it back to the offshore platform, where the \ce{CO2} gas is transported and stored in the oil reservoir where the natural gas originated from. This project will look at the feasibility of an offshore blue hydrogen production where the natural gas from the oil reservoir will be extracted and reform into mainly hydrogen and \ce{CO2} gas on the same unit, either on a ship or a platform. The \ce{CO2} will be captured on the same unit and transported and stored in the oil reservoir. The hydrogen gas will be transported to an onshore plant. If a such plant is feasible, where hydrogen can be produced without large amounts of emission, this project could be a relatively large step towards the green change. 

There is a lot of research needed to study if a project of this scale is feasible. There is the model feasibility and the economic feasibility. This project is a part of the SUBPRO Zero Blue Hydrogen project, which cooperating with the oil and gas industry in Norway. The goal is develop the exiting steady-state model at SUBPRO by implementing aspects that reflects more to a real-life plant that can be realized. 