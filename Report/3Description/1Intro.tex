Even though there are many ways that are well studied for blue hydrogen production, there are other circumstances when designing a plant on offshore. There are smaller constraints for space and weight, and the building cost will be greater. An ordinary blue hydrogen production plant uses a large furnace to perform the steam methane reforming, which takes up a lot of space. Instead of taking basis on a regular steam methane reformer, the project will take basis on a method that a company named \textit{Johnson Matthey} has designed. This method replaces the large furnace with an autothermal reformer (ATR) combined with the gas heated reformer (GHR) with pure oxygen stream into the ATR. This method has reported high hydrogen purity with a low carbon emission. A detailed description of the process will be broken down into parts and described in the next subsection. 